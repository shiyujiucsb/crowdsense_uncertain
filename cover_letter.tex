\documentclass{article}

\usepackage{setspace}
\usepackage{fullpage}
\usepackage{color}

\begin{document}

\setlength{\parindent}{0cm}
\setlength{\parskip}{\baselineskip}

Dear Editors,

We are submitting our paper \emph{Incentive Mechanisms for Crowdsourcing to Smartphones with Uncertain Sensing Time} as a transaction paper to \emph{IEEE Transactions on Mobile Computing}.

This paper is the continuation of our conference paper \emph{Crowdsourcing with Trembles: Incentive Mechanisms for Mobile Phones with Uncertain Sensing Time}, which was accepted by \emph{IEEE ICC 2015}.

This submission contains significant differences and extra contributions compared to the conference paper. We summarize the differences as following.
\begin{itemize}
\item We have added Section III to evaluate the negative impacts of sensing time uncertainty on crowdsourcing incentive mechanisms. The evaluation leverages simulations and the results are presented as Fig. 3 and 4. Based on the results, we conclude that the uncertainty perturbation has significant negative impacts and hence it is worth continuing our subsequent discussions.
\item We have contributed the solution to maximize the platform utility by strategizing the reward. We have proposed Algorithm 3 to strategize the reward with respect to the platform, and we have also shown the existence of the reward solution by Theorem 6. A new Section V has been added to give a detailed discussion about Algorithm 3.
\item We have considered the scenario of sequential games, and a new Algorithm 4 has been proposed to estimate the maximum uncertainty perturbation. A new Theorem 5 has shown that our estimation has no bias. We have also given detailed discussions about the complexity, adaptiveness and interactiveness of Algorithm 4.
\item We have updated the simulation results in Section VII in order to verify the correctness and effectiveness of our added algorithms in Sections V, VI, etc. We have given the extended simulation results as Fig. 7, 9, 10, 11, 12, 13 and related discussions in Section VII.
\item We have added the proofs and detailed discussions for all the theoretic results, including all the theorems and lemmas in the conference version. Our treatment also contains many new results including Theorem 2, 5, 6 and Lemma 2, 3, 4 in this submission.
\item We have also added the appendix to discuss our assumptions on the probability distribution of the sensing perturbations, and more detailed reasonings. In the appendix we give detailed theoretic reasoning and rigorously show that the mean of the perturbation is approximately the middle point of the support interval. This result gives rise to our discussions in the body texts.
\item We have revised the Abstract, Sections I, IX, etc. (i.e., Fig. 1 has been added to help illustrate the basic idea of this paper) to reflect our new contributions, and we have given more discussions about some notions and conceptions (i.e., revenue coefficient, maximum sensing perturbation, etc.) in the introduction part. Table 1 has been given to make the notations clear.
\end{itemize}

With all the above revisions, we believe the difference between the papers is more than 30\%.

Thank you very much for your consideration.

Shiyu Ji\\
Department of Computer Science\\
Oklahoma State University
\end{document}